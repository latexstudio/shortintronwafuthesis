\usetheme[
%%% 外部主题选项
%    hidetitle,           % 隐藏边栏中的短标题
%    hideauthor,          % 隐藏边栏中的作者缩写
%    hideinstitute,       % 隐藏边栏底部的单位缩写
%    shownavsym,          % 显示导航符号
    width=1.8cm,           % 边栏宽度 (默认是 2 cm)
%    hideothersubsections,% 除了当前section的subsection隐藏其它所有 subsections
%    hideallsubsections,  % 隐藏所有 subsections
    left,               % 边栏位置 (默认在右边)
%%% 颜色主题选项
    %lightheaderbg       % 页眉背景颜色
  ]{NWSUAFsidebar}
  
\definecolor{Descitem}{RGB}{0, 0, 139}

\definecolor{StdTitle}{RGB}{26, 33, 141}
\definecolor{StdBody}{RGB}{213,24,0}

\definecolor{AlTitle}{RGB}{255, 190, 190}
\definecolor{AlBody}{RGB}{213,24,0}

\definecolor{ExTitle}{RGB}{201, 217, 217}
\definecolor{ExBody}{RGB}{213,24,0}  
  
% Standard block
\setbeamercolor{block title}{fg = Descitem, bg = StdTitle!15!white}
\setbeamercolor{block body}{bg = StdBody!5!white}
% Alert block

\setbeamercolor{block title alerted}{bg = AlTitle}
\setbeamercolor{block body alerted}{bg = AlBody!5!white}
% Example block

\setbeamercolor{block title example}{bg = ExTitle}
\setbeamercolor{block body example}{bg = ExBody!5!white}

\setbeamerfont{block title}{size=\scriptsize}
  
\setbeamertemplate{blocks}[rounded][shadow=true]

\setbeamerfont{footnote}{size=\zihao{7}} % 改变脚注字号

% 设置minted宏包编排代码的参数及用于latex代码排版的简化命令
\definecolor{listinggray}{gray}{0.92}
\setminted{fontsize=\tiny, mathescape, breaklines=true, breakautoindent=false, autogobble}
\newmintinline{tex}{fontsize=\normalsize}
\newmintinline[texinlinett]{tex}{fontsize=\normalsize,escapeinside=||}
\newminted{tex}{bgcolor=listinggray, frame=lines}
\newminted[texcodett]{tex}{bgcolor=listinggray, frame=lines, escapeinside=||}
\newminted[shell]{sh}{autogobble,fontsize=\small,frame=lines}
\newmintedfile{tex}{bgcolor=listinggray, linenos=true, frame=lines}

% 定义颜色
% ==================================================
\definecolor{mygray}{gray}{.9}
\definecolor{mypink}{rgb}{.99,.91,.95}
\definecolor{mycyan}{cmyk}{.3,0,0,0}

% 改变脚注的符号
\setbeamerfont{footnote}{size=\zihao{7}} % 改变脚注字号
\makeatletter
\def\@fnsymbol#1{\ensuremath{\ifcase#1\or *\or \dagger\or \ddagger\or
   \mathsection\or \mathparagraph\or \|\or **\or \dagger\dagger
   \or \ddagger\ddagger \else\@ctrerr\fi}}
\makeatother
\renewcommand{\thefootnote}{\fnsymbol{footnote}}

%% 自定义相关的名称宏命令
%% ==================================================
%% \newcommand{\yourcommand}[参数个数]{内容}
% 西北农林科技大学各单位名称
\newcommand{\nwsuaf}{西北农林科技大学}
\newcommand{\cie}{信息工程学院}
\newcommand{\cs}{计算机科学系}

% 定义引号命令
\newcommand{\qtmark}[1]{``#1''}%``''
% 定义带引号的加粗强调命令
\newcommand{\qtb}[1]{\qtmark{\emph{#1}}}
% 定义带引号的加粗加红强调命令
\newcommand{\qtbr}[1]{\qtmark{\emph{\alert{#1}}}}

% 自定义latex简单教程中要用到的宏命令
\newcommand\tex{{\fontfamily{cmr}\selectfont \TeX}}
\newcommand\latex{{\fontfamily{cmr}\selectfont \LaTeX}}
\newcommand\latextwoe{{\fontfamily{cmr}\selectfont \LaTeX2e}}
\newcommand\latexpdf{\texorpdfstring{\latex{}}{LaTeX}}
\newcommand\msoffice{{\rmfamily MS Office}}
\newcommand\msofficepdf{\texorpdfstring{\msoffice{}}{MS Office}}

\newcommand\demo[1][]{\href{#1}{\ttfamily \textless demo\textgreater}}
\newcommand\wysiwym{\textsc{WYSIWYM}---所想即所得}
\newcommand\wysiwyg{\textsc{WYSIWYG}---所见即所得}
\newcommand\myrule{\rule{\textwidth}{.5pt}}

\colorlet{msofficecolour}{red!90!white}
\colorlet{latexcolour}{green!90!black}

\newcommand\latexc{\textcolor{white}{\latex}}
\newcommand\msofficec{\textcolor{white}{\msoffice}}

% 定义TeXLive的LOGO
\renewcommand*\sfdefault{uop}
\renewcommand*\familydefault{\sfdefault}
%
\definecolor{tlblue}{HTML}{0078B8}

\newcommand*\TeXLive{T\kern -.1667em\lower .5ex\hbox {E}\kern
  -.025emX\,Live}

\newcommand\tlive[1][2018]{
  \begin{tikzpicture}[x=1pt,y=1pt,inner sep=0pt,outer sep=0pt]
    \fill [tlblue] (0,0) rectangle (567,160);
    \node [white] at (29.7,33.8) [anchor=south west]
    {\scalebox{10}{\bfseries\TeXLive\~ #1}};
    \node at (388,9) [anchor=south west] {\includegraphics[width=15em]{tl-lion}};
    % \node [anchor=south west] {\includegraphics[height=16em]{logo}};
  \end{tikzpicture}%
}

%叉号与对号
\newcommand{\goodmark}{\textcolor{green!50!black}{\Pisymbol{pzd}{52}}}
\newcommand{\badmark}{\textcolor{red}{\Pisymbol{pzd}{56}}}

% 定义罗马数字
\makeatletter
\newcommand{\rmnum}[1]{\romannumeral #1}
\newcommand{\Rmnum}[1]{\expandafter\@slowromancap\romannumeral #1@}
\makeatother

% 定义破折号
\newcommand{\pozhehao}{\kern0.3ex\rule[0.8ex]{2em}{0.1ex}\kern0.3ex}

% 用于绘制geometry页面图
\newcommand\gpart[1]{\textsf{\textsl{\color[rgb]{.0,.45,.7}#1}}}%
\def\Gm{\textsf{geometry}}
  
% 超链接的颜色
\newcommand{\chref}[2]{%
  \href{#1}{{\usebeamercolor[bg]{NWSUAFsidebar}#2}}%
}

% 定义积分中的微分d为直立体,并与变量分开一点距离\!
\DeclareMathOperator\dif{\mathrm{d}\!}

% 输出数学符号表的命令---摘自lshort
% 公式输出
\newcommand{\X}[1]{$#1$&\texttt{\string#1}\hspace*{1ex}}
% 文本输出 
\newcommand{\SC}[1]{#1&\texttt{\string#1}\hspace*{1ex}}
% 文本模式的重音符号
\newcommand{\A}[1]{#1&\texttt{\string#1}\hspace*{1ex}}
\newcommand{\BB}[2]{#1#2&\texttt{\string#1{} #2}\hspace*{1ex}}

\newcommand{\W}[2]{$#1{#2}$&
  \texttt{\string#1}\texttt{\string{\string#2\string}}\hspace*{1ex}}
\newcommand{\Y}[1]{$\big#1$ &\texttt{\string#1}}  %
% 数学符号表格环境
\newsavebox{\symbbox}
\newenvironment{symbols}[1]%
{\par\vspace*{2ex}
\renewcommand{\arraystretch}{1.1}
\begin{lrbox}{\symbbox}
\hspace*{4ex}\begin{tabular}{@{}#1@{}}\hline}%
{\\ \hline \end{tabular}\end{lrbox}\makebox[\textwidth]{\usebox{\symbbox}}\par\medskip}

% 定义子公式编号用的环境
\newenvironment{mysubeqn}%
          {\begin{subequations}
              \renewcommand\theequation{\theparentequation-\roman{equation}}}%
            {\end{subequations}}
% ==================================================
\newcommand{\nwafuthesis}{%
  \makebox{\rmfamily%
    N\hspace{-0.2ex}\raisebox{-0.5ex}{W}\raisebox{0.5ex}{\hspace{-0.2ex}\textsc{afu}}\hspace{0.3ex}%
    \textsc{Thesis}}}          

%% 定义自动扩展垂直间距的命令\stretchon和\stretchoff
%% ==================================================
\def\itemsymbol{$\blacktriangleright$}
\let\svpar\par
\let\svitemize\itemize
\let\svenditemize\enditemize
\let\svitem\item
\let\svcenter\center
\let\svendcenter\endcenter
\let\svcolumn\column
\let\svendcolumn\endcolumn
\def\newitem{\renewcommand\item[1][\itemsymbol]{\vfill\svitem[##1]}}%
\def\newpar{\def\par{\svpar\vfill}}%
\newcommand\stretchon{%
  \newpar%
  \renewcommand\item[1][\itemsymbol]{\svitem[##1]\newitem}%
  \renewenvironment{itemize}%
    {\svitemize}{\svenditemize\newpar\par}%
  \renewenvironment{center}%
    {\svcenter\newpar}{\svendcenter\newpar}%
  \renewenvironment{column}[2]%
    {\svcolumn{##1}\setlength{\parskip}{\columnskip}##2}%
    {\svendcolumn\vspace{\columnskip}}%
}
\newcommand\stretchoff{%
  \let\par\svpar%
  \let\item\svitem%
  \let\itemize\svitemize%
  \let\enditemize\svenditemize%
  \let\center\svcenter%
  \let\endcenter\svendcenter%
  \let\column\svcolumn%
  \let\endcolumn\svendcolumn%
}

%% 签署春秋学期日期命令
\newcommand{\tomonth}{
  \the\year 年\the\month 月
}


\newcommand{\tomonthen}{
  \ifcase\the\month
  \or January%
  \or February%
  \or March%
  \or April%
  \or May%
  \or June%
  \or July%
  \or August%
  \or September%
  \or October%
  \or November%
  \or December%
  \fi, \the\year
}

\newcommand{\tosemester}{
  \the\year 年\ 
  \ifcase\the\month
  \or 秋%
  \or 春%
  \or 春%
  \or 春%
  \or 春%
  \or 春%
  \or 春%
  \or 夏%
  \or 秋%
  \or 秋%
  \or 秋%
  \or 秋%
  \fi 
}

\newcommand{\tosemesteren}{  
  \ifcase\the\month
  \or Autumn%
  \or Spring%
  \or Spring%
  \or Spring%
  \or Spring%
  \or Spring%
  \or Summer%
  \or Autumn%
  \or Autumn%
  \or Autumn%
  \or Autumn%
  \or Autumn%
  \fi, \the\year
}

%% 分栏宽度
\newlength\columnskip
\columnskip 0pt
%% ==================================================          
          
% TiKz绘图设置
\usetikzlibrary{arrows}
% TikZ宏包扩展
\usetikzlibrary{positioning}
\usetikzlibrary{decorations}
% pgfplots设置
\pgfplotsset{compat=newest,compat/show suggested version=false}

% 插图路径设置
% ==================================================
\graphicspath{{figures/}}%图片所在的目录
% ==================================================

% 为标题页指定一个 logo
\pgfdeclareimage[height=0.8cm]{titlepagelogo}{nwsuaflogo/nwsuaf_logo_new}% 标题页
\titlegraphic{% 标题页底部
  \pgfuseimage{titlepagelogo}%
}

% 自动插入的帧
% ==================================================
% ==================================================
% 每一讲前面添加的帧
% ==================================================
% \AtBeginLecture{
%   \begin{frame}[allowframebreaks]{本讲主要内容}{目录}
%     %\Large
%     %\centering
%     %\insertlecture
%     \tableofcontents
%   \end{frame}
% }

% \iffalse
% \AtBeginSection[]{ \frame{
%     \footnotesize
%     \frametitle{本讲的主要内容}
%     \tableofcontents[currentsection]
%   }
% }
% %\iffalse
% \AtBeginSubsection[]
% {
%   \begin{frame}
%     \footnotesize
%     \frametitle{本讲的主要内容}
%     \tableofcontents[currentsection,currentsubsection]
%   \end{frame}
% }
% \fi

%%% Local Variables: 
%%% mode: latex
%%% TeX-master: "../main.tex"
%%% End: 
